\documentclass{article}

\def\TemplatePath{./}

%%% Template Packages %%%

\usepackage{xifthen} % Template stuff 
\usepackage{graphicx} % Images
\usepackage{tcolorbox} % Color Box
\usepackage[%
vmargin=2cm,%
hmargin=2.25cm,%
headheight=22pt%
]{geometry} % Page Geometry
\usepackage{fancyhdr} % Header / Footer Styles
\usepackage{extramarks} % Header / Footer Marks

\usepackage{ragged2e} % Text Align
\usepackage{amsmath} % Math Align

\usepackage{booktabs} % Tables+ Package
\usepackage[inline]{enumitem} % Enumarete+ Package
\usepackage{xparse} % Command Args Split Package

\usepackage{mathtools} % Math General
\usepackage{amsthm} % Math Envs
\usepackage{unicode-math} % Math Symbols

\usepackage{polyglossia} % Language
    \setdefaultlanguage{spanish}%

%%% Variables and Such %%%
\ifthenelse{\isundefined{\TemplatePath}}{%
    \def\TemplatePath{template/}
}{}

%%% Template Styles %%%

% Header / Footer Styles
\pagestyle{fancy}
\RenewDocumentCommand{\headrule}{}{%
    \rule[0.1cm]{\textwidth}{0.1mm}%
}

\ifthenelse{\isundefined{\MyTitle}}{}{%
\fancyhf[HC]{{\slshape \MyTitle{}}}
}%
\fancyhead[HL]{\firstleftxmark}
\fancyhead[HR]{\lastleftxmark}

\RenewDocumentCommand{\footrule}{}{%
    \rule[0.1cm]{\textwidth}{0.1mm}%
}

\renewcommand{\thefootnote}{\Roman{footnote}} % Changing footnotes arabic to roman numbers

% Redefine \maketitle
\RenewDocumentCommand{\maketitle}{s}{%
    \begin{@twocolumntrue}%
        \begin{minipage}{0.3\textwidth}%
            \begin{Center}
                \includegraphics[width=0.5\textwidth]{\TemplatePath src/unal_logo.pdf}%
            \end{Center}
        \end{minipage}%
        \begin{minipage}{0.7\textwidth}{%
            \begin{Center}
                \ifthenelse{\isundefined{\MyClass}}{%
                    {\large Set \textbackslash{}def\textbackslash{}MyClass\{your\_class\} in the preamble} \\[0ex]%
                }{%

                    {\large \itshape \MyClass{}} \\[0.75ex]
                }%
                \ifthenelse{\isundefined{\MyTitle}}{%
                    {\huge Set \textbackslash{}def\textbackslash{}MyTitle\{doc\_title\} in the preamble} \\[4ex]%
                }{%
                    {\huge \slshape \MyTitle{}} \\[1.5ex]%
                }%
                \ifthenelse{\isundefined{\MyAuthor}}{%
                    {\Large Set \textbackslash{}def\textbackslash{}MyAuthor\{your\_name\} in the preamble} \\%
                }{%
                    {\Large \MyAuthor{}} \\%
                }%
                \ifthenelse{\isundefined{\MyEmail}}{%
                    {\small Set \textbackslash{}def\textbackslash{}MyEmail\{your\_email\} in the preamble} \\[4ex]%
                }{%
                    {\small \MyEmail{}} \\[1ex]%
                }%
                \ifthenelse{\isundefined{\MyDate}}{%
                    Set \textbackslash{}def\textbackslash{}MyDate\{doc\_date\} in the preamble%
                }{%
                    \MyDate{}%
                }%
            \end{Center}
        }%
        \end{minipage}%
    \end{@twocolumntrue}%
    \vspace{0.25cm}%
    \begin{Center}%
        \rule[0cm]{\textwidth}{0.1mm}%
    \end{Center}%
}

% Enumerate Lists types
\newcommand\SetItemnumber[1]{\setcounter{enumi}{\numexpr#1-1\relax}}
\newcommand{\listSubscript}[2]{\(#1_{#2}\)}
\newcommand{\listAlph}{\alph*.}
\newcommand{\listALPH}{\Alph*.}

% Fast Supper and Sub -scripts
\NewDocumentCommand{\tsup}{m} {\textsuperscript{#1}}
\NewDocumentCommand{\tsub}{m} {\textsubscript{#1}}

% Set of Numbers
\def\realR{\symbb{R}} % Real
\def\realN{\symbb{N}} % Natural
\def\realZ{\symbb{Z}} % Integers
\def\realQ{\symbb{Q}} % Rational
\def\realC{\symbb{C}} % Complex
\def\realC{\symbb{I}} % Irrational

%%% Template Math stuff %%%

% Cases and stuff
\newtheorem{TMPMathCases}[]{Caso}
\NewDocumentCommand{\ResetCases}{} {\setcounter{TMPMathCases}{0}}
\NewDocumentEnvironment{mathcase}{O{}m} {%
    \IfValueTF{#1}{%
        \begin{TMPMathCases}[#1] % 
            #2%
        \end{TMPMathCases}%
    }{%
        \begin{TMPMathCases} % 
            #2%
        \end{TMPMathCases}%
    }
} {}

% Fast Function
\NewDocumentCommand{\ffunc}{mm} {%
    #1\left(#2\right)
}

% Fast Set
\NewDocumentCommand{\fset}{m} {%
    \left\{%
        #1 %
    \right\}%
}


%\usepackage{biblatex}
%\addbibresource{src/references.bib}

\def\MyClass{Introducción a la Teoria de Conjuntos}
\def\MyTitle{Examen 2 -- Punto 3}
\def\MyAuthor{Martín Steven Hernández Ortiz}
\def\MyEmail{mahernandezor@unal.edu.co}
\def\MyDate{\today}

\begin{document}
\maketitle

\begin{itemize} 
    \item Muestre de manera explicita un subconjunto de \(\realR{}\) con el orden usual de \(\realR{}\) \(\left(<\right)\) que tenga el mismo tipo de orden que \(\omega^{3}\). \\
        Debemos mostrar que existe un \(R \subset \realR\) tal que exista un isomorfismo de orden entre \(\omega^3\) y \(R\).
        \\
        \textbf{Respuesta: } 
        Tomando la serie geometrica 
        \(
            \sum_{n=0}^{\infty} ar^n = \frac{a}{1-r}
        \), 
        donde \(\left|r\right| < 1\) y \(k \in \realR\). Tal que para, \(k = 1, r = \frac{1}{2}\), se tiene que \(\sum_{n=0}^{\infty}\frac{1}{2^n} = \frac{1}{1 - \frac{1}{2}} = 2\).
        Ahora, sean las funciones \(f_1: \realN \rightarrow \realR, f_2: \realN \rightarrow \realR, f_3: \realN \rightarrow \realR\), tales que 
        \[
            \begin{aligned}
                f_1(a) = \sum_{n=0}^{a} \left(\frac{1}{2}\right){}^{n} &
                \hspace{0.5cm}
                f_2(b) = \sum_{n=0}^{b} 3\left(\frac{1}{2}\right){}^{n} &
                \hspace{0.5cm}
                f_3(c) = \sum_{n=0}^{c} 10\left(\frac{1}{2}\right){}^{n}
            \end{aligned}
        \]
        Veamos que para \(i = 1, 2, 3\), si \(d_1, d_2 \in \realN\) tales que \(d_1 < d_2\), se va a tener que \(f_i(d_1) < f_i(d_2)\).
        Luego, como \(0 = \text{mín}\left(\realN\right)\), vamos a tener que para todo \(d \in \realN\), \(f_i(0) \leq f_i(d)\). 
        Además también se va a tener que 
        \[
            \begin{aligned}
                \lim_{a \rightarrow \infty} f_1(a) = \lim_{a \rightarrow \infty}\sum_{n=0}^{a} \frac{1}{2^{n}} = 2&
                \hspace{0.25cm}
                \lim_{b \rightarrow \infty} f_2(b) = \lim_{b \rightarrow \infty}\sum_{n=0}^{b} 3\left(\frac{1}{2}\right){}^{n} = 6
                \hspace{0.25cm}
                \lim_{c \rightarrow \infty} f_3(c) = \lim_{c \rightarrow \infty}\sum_{n=0}^{c} 10\left(\frac{1}{2}\right){}^{n} = 20
            \end{aligned}
        \]
        Entonces vamos a poder acotar cada función, tal que para todo \(m \in \realN\) 
        \[
            \begin{aligned}
                1 \leq f_1(m) < 2 &
                \hspace{0.5cm}
                3 \leq f_2(m) < 6 &
                \hspace{0.5cm}
                10 \leq f_3(m) < 20
            \end{aligned}
        \]
        Ahora, veamos que para todo \(m' \in \realN\) tal que \(m' > 0\),
        \[
            \begin{aligned}
                14 < f_1(m') + f_2(0) + f_3(0) < 15, &
                \hspace{0.5cm}
                15 < f_1(0) + f_2(m') + f_3(0) < 17, \\
                \hspace{0.5cm}
                17 < f_1(m') + f_2(m') + f_3(0) < 18, &
                \hspace{0.5cm}
                18 < f_1(0) + f_2(0) + f_3(m') < 23, \\
                \hspace{0.5cm}
                23 < f_1(m') + f_2(0) + f_3(m') < 25, &
                \hspace{0.5cm}
                25 < f_1(0) + f_2(m') + f_3(m') < 27, \\
            \end{aligned}
        \]
        \vspace{-0.25cm}
        \[
                \text{y } 27 < f_1(m') + f_2(m') + f_3(m') < 28
        \]
        Ahora, definamos al conjunto \(R\), como \(R := \left\{f_1(i) + f_2(j) + f_3(l): i, j, l \in \realN \right\}\). 
        Es sencillo ver que \(R \subset \realR\) ya que \(\text{Im}(f_i) \subset \realR\) para \(i = 1, 2, 3\). 
        Ahora, sea una función \(\phi: \omega^3 \rightarrow R\), definida como
        \[
            \phi(\alpha, \beta, \gamma) = f_1(\alpha) + f_2(\beta) + f_3(\gamma)
        \]
        veamos que \(\phi\) es un isomorfismo entre \(\omega^3\) y \(R\).
        \ResetCases{}
        \begin{mathcase}{\(\phi\) es un homomorfismo de orden entre \(\omega^3\) y \(R\)}
            \vspace{-0.4cm}
            Primero, veamos que \(\phi\) es una función creciente, sean ordinales cualesquiera tales que,
            \((\alpha_1, \beta_1, \gamma_1) < (\alpha_2, \beta_2, \gamma_2)\). Veamos que, como \(f_1, f_2, f_3\) son funciones crecientes, 
            \[
                \begin{aligned}
                    \phi(\alpha_2, \beta_2, \gamma_2) - \phi(\alpha_1, \beta_2, \gamma_1) &=
                    (f_1(\alpha_2) + f_2(\beta_2) + f_3(\gamma_2)) + (- f_1(\alpha_1) - f_2(\beta_1) - f_3(\gamma_1)) \\
                    &= (f_1(\alpha_2) - f_1(\alpha_1)) + (f_2(\beta_2) - f_2(\beta_1)) + (f_3(\gamma_2) - f_3(\gamma_1)) \\
                    &> 0 + 0 + 0 = 0
                \end{aligned}
            \]
            Como \(\phi(\alpha_2, \beta_2, \gamma_2) - \phi(\alpha_1, \beta_2, \gamma_1) > 0\), entonces \(\phi(\alpha_2, \beta_2, \gamma_2) > \phi(\alpha_1, \beta_2, \gamma_1)\).
            De manera analoga se puede probar que \(phi\) es un homomorfismo de orden entre \(\omega^3\) y \(R\).
        \end{mathcase}
        \begin{mathcase}{\(\phi\) es inyectiva}
            \vspace{-0.4cm}
            Si 
            \(
                (\alpha_1, \beta_1, \gamma_1) \neq (\alpha_2, \beta_2, \gamma_2) 
            \), 
            sin perdida de la generalidad, veamos que \((\alpha_1, \beta_1, \gamma_1) < (\alpha_2, \beta_2, \gamma_2)\), 
            entonces, como \(f_1, f_2\), \(f_3\) y \(\phi\) son crecientes,
            se tiene que \(\phi(\alpha_1, \beta_1, \gamma_1) < \phi(\alpha_2, \beta_2, \gamma_2)\), 
            y \(\phi(\alpha_1, \beta_1, \gamma_1) \neq \phi(\alpha_2, \beta_2, \gamma_2)\).
            Es decir, \(\phi\) es inyectiva
        \end{mathcase}
        \begin{mathcase}{\(\phi\) es sobreyectiva}
            \vspace{-0.4cm}
            Tome cualquier \(r \in R\), por definición de \(R\) existen \(i, j, l \in \realN\) tales que \(r = f_1(i) + f_2(j) + f_3(l)\),
            como \(\realN = \omega\), entonces \(i, j, l \in \omega\). Luego, \((i,j,l) \in \omega^3\) y por definición de \(phi\), 
            \(\phi(i, j, l) = f_1(i) + f_2(j) + f_3(l) = r\). Es decir, \(\phi\) es sobreyectiva.
        \end{mathcase}
        Concluyendo que \(\phi\) es un isomorfismo de orden entre \(\omega^3\) y \(R\), donde \(R \subset \realR\).
\end{itemize}

\end{document}
